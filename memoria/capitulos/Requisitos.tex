\chapter{Análisis y especificación de requisitos}

En este capitulo se trata de comprender y definir las necesidades, objetivos, funcionalidades y actores que van a hacer uso de la herramienta a desarrollar. Para lograrlo se van plantear diferentes requisitos funcionales, no funcionales y casos de uso como solución a las siguientes preguntas.

- ¿Cómo debe comportarse la aplicación?

- ¿Qué flujo tiene cada funcionalidad?

- ¿Qué limitaciones y restricciones existen?

- ¿Qué tipos de usuarios existen?

- ¿Qué acciones puede realizar cada tipo de usuario?

- ¿Cómo satisfacer las necesidades del 
usuario?

- ¿Qué espera el usuario?


%\begin{itemize}
%    \item ¿Cómo debe comportarse la aplicación?
%    \item ¿Qué flujo tiene cada funcionalidad?
%    \item ¿Qué limitaciones y restricciones existen?
%    \item ¿Qué tipos de usuarios existen?
%    \item ¿Qué acciones puede realizar cada tipo de usuario?
%    \item ¿Cómo satisfacer las necesidades del usuario?
%    \item ¿Qué espera el usuario?
%\end{itemize}


\section{Especificación de requisitos}

\subsection{Requisitos funcionales}

Los requisitos funcionales (R.F.) son aquellos que definen como un sistema o componente tiene que funcionar o comportarse. Estos en este caso,
son los siguientes:

\begin{itemize}
    \item R.F. 1: El sistema permitirá representar información a través de distintas visualizaciones.
    \item R.F. 2: Las visualizaciones serán interactivas cuando el usuario hace click o hover.
    \item R.F. 3: El usuario podrá filtrar los datos de entrada de las visualizaciones.
    \item R.F. 4: El sistema permitirá mostrar información relevante a un paciente específico.
    \item R.F. 5: El sistema utlilizará Inteligencia Artificial para generar resúmenes del historial de un paciente.
    \item R.F. 6: El sistema implementará un chat con Inteligencia Artificial para que el usuario realice preguntas sobre los datos.
    
\end{itemize}

\subsection{Requisitos no funcionales}

Los requisitos no funcionales (R.N.F.) son aquellas restricciones que se imponen o existen en relación a la distinta funcionalidad que existe en la herramienta, es decir, las restricciones que presentan los requisitos funcionales. Estos en este caso, son los siguientes:

\begin{itemize}
    \item R.N.F. 1: El sistema en caso de error, gestionará este de manera interna para no afectar la experiencia del usuario que este consumiendo la herramienta.
    \item R.N.F. 2: La herramienta será compatible con los navegadores más usados en la actualidad, así como su uso en distintos sistemas operativos. 
    \item R.N.F. 3: El código desarrollado tendrá que estar bien modularizado y estructurado para que su mantenimiento sea una tarea lo más sencilla posible.
    
\end{itemize}

%\section{Casos de uso}
%\subsection{Actores}

\section{Actores}

Los actores son aquellas personas o entidades que usan la herramienta. Estos son fundamentales para la creación de casos de uso, ya que son aquellos que desde su perspectiva muestran como el sistema se comporta. 

En el contexto de este proyecto se ha identificado un único actor humano principal: el Usuario. 

\begin{table}[H]
    \centering
    \begin{tabular}{|l|p{11cm}|}
        \hline
        \textbf{Nombre} & Usuario \\
        \hline
        \textbf{Identificador} & ACT-1 \\
        \hline
        \textbf{Descripción} & Persona que utiliza la aplicación web para consultar estadísticas, visualizar gráficos, buscar pacientes concretos y obtener apoyo de IA (resúmenes y chat) sobre los datos de MIMIC-IV. \\
        \hline
        \textbf{Características} & Perfil clínico, investigador o estudiante; acceso vía navegador; no requiere conocimientos de programación ni de consultas sobre la base de datos; usa la interfaz para filtrar, explorar y comprender la información. \\
        \hline
        \textbf{Referencias} & Requisitos funcionales: RF1, RF2, RF3, RF4, RF5, RF6. \\
        \hline
        \textbf{Versión} & 1.0 \\
        \hline
    \end{tabular}
    \caption{Actor principal del sistema.}
\end{table}

%\subsection{Diagramas}

\section{Casos de Uso}

Un caso de uso es aquel que define como ante una situación o funcionalidad, tanto la herramienta como el sistema se tiene que comportar y actuar. Estos vienen definidos por los requisitos funcionales, por lo que cada uno de ellos está relacionado con uno o varios de estos requisitos.


\begin{table}[H]
    \centering
    \begin{tabular}{|l|p{11cm}|}
        \hline
        \textbf{Nombre} & CU-1: Acceder al dashboard \\
        \hline
        \textbf{Actores} & ACT-1 Usuario \\
        \hline
        \textbf{Tipo} & Primario, básico y esencial \\
        \hline
        \textbf{Referencias} & RF1 \\
        \hline
        \textbf{Precondición} & Backend disponible; datos precalculados accesibles. \\
        \hline
        \textbf{Poscondición} & El usuario visualiza KPIs y accesos a visualizaciones. \\
        \hline
        \textbf{Propósito} & Ofrecer una visión global rápida del estado de la base de datos. \\
        \hline
        \textbf{Resumen} & El sistema recupera estadísticas y las muestra en el dashboard. \\
        \hline
    \end{tabular}
\end{table}

\begin{table}[H]
    \centering
    \begin{tabular}{|l|p{11cm}|}
        \hline
        \textbf{Nombre} & CU-2: Explorar una visualización con filtros \\
        \hline
        \textbf{Actores} & ACT-1 Usuario \\
        \hline
        \textbf{Tipo} & Primario, básico y esencial \\
        \hline
        \textbf{Referencias} & RF1, RF2, RF3 \\
        \hline
        \textbf{Precondición} & Endpoint del gráfico disponible. \\
        \hline
        \textbf{Poscondición} & Gráfico renderizado con los filtros aplicados. \\
        \hline
        \textbf{Propósito} & Permitir exploración interactiva ajustando parámetros de visualización. \\
        \hline
        \textbf{Resumen} & El usuario selecciona la visualización y ajusta filtros; el sistema actualiza el gráfico. \\
        \hline
    \end{tabular}
\end{table}

\begin{table}[H]
    \centering
    \begin{tabular}{|l|p{11cm}|}
        \hline
        \textbf{Nombre} & CU-3: Buscar un paciente por identificador \\
        \hline
        \textbf{Actores} & ACT-1 Usuario \\
        \hline
        \textbf{Tipo} & Primario, básico \\
        \hline
        \textbf{Referencias} & RF4 \\
        \hline
        \textbf{Precondición} & El identificador \texttt{subject\_id} es válido o comprobable. \\
        \hline
        \textbf{Poscondición} & Navegación a la ficha del paciente o aviso de no encontrado. \\
        \hline
        \textbf{Propósito} & Acceder rápidamente a la información de un paciente concreto. \\
        \hline
        \textbf{Resumen} & El sistema verifica la existencia del paciente y redirige a su ficha si existe. \\
        \hline
    \end{tabular}
\end{table}

\begin{table}[H]
    \centering
    \begin{tabular}{|l|p{11cm}|}
        \hline
        \textbf{Nombre} & CU-4: Consultar la ficha de un paciente \\
        \hline
        \textbf{Actores} & ACT-1 Usuario \\
        \hline
        \textbf{Tipo} & Primario, esencial \\
        \hline
        \textbf{Referencias} & RF4 \\
        \hline
        \textbf{Precondición} & Paciente existente. \\
        \hline
        \textbf{Poscondición} & Información consolidada visible por ingreso hospitalario. \\
        \hline
        \textbf{Propósito} & Presentar el historial clínico estructurado del paciente. \\
        \hline
        \textbf{Resumen} & El sistema obtiene y muestra datos demográficos, ingresos, diagnósticos, procedimientos y laboratorio. \\
        \hline
    \end{tabular}
\end{table}

\begin{table}[H]
    \centering
    \begin{tabular}{|l|p{11cm}|}
        \hline
        \textbf{Nombre} & CU-5: Obtener resumen con IA del historial \\
        \hline
        \textbf{Actores} & ACT-1 Usuario \\
        \hline
        \textbf{Tipo} & Primario, básico \\
        \hline
        \textbf{Referencias} & RF5 \\
        \hline
        \textbf{Precondición} & Servicio de IA accesible; datos del paciente cargados en la vista. \\
        \hline
        \textbf{Poscondición} & Resumen clínico generado y visible en la interfaz. \\
        \hline
        \textbf{Propósito} & Facilitar una comprensión rápida del historial del paciente. \\
        \hline
        \textbf{Resumen} & El sistema envía los datos del paciente al servicio de IA y muestra el resumen devuelto. \\
        \hline
    \end{tabular}
\end{table}

\begin{table}[H]
    \centering
    \begin{tabular}{|l|p{11cm}|}
        \hline
        \textbf{Nombre} & CU-6: Realizar consulta en lenguaje natural (chat) \\
        \hline
        \textbf{Actores} & ACT-1 Usuario \\
        \hline
        \textbf{Tipo} & Primario, opcional \\
        \hline
        \textbf{Referencias} & RF6 \\
        \hline
        \textbf{Precondición} & Servicio de IA y herramientas MCP operativas. \\
        \hline
        \textbf{Poscondición} & Respuesta contextual mostrada al usuario. \\
        \hline
        \textbf{Propósito} & Permitir consultas ad-hoc sobre los datos en lenguaje natural. \\
        \hline
        \textbf{Resumen} & El sistema procesa la pregunta usando IA y datos disponibles y devuelve la respuesta. \\
        \hline
    \end{tabular}
\end{table}


% actores, casos de uso, diagramas de actividad , etc ..... ??? 