\chapter{Análisis y especificación de requisitos}

En este capitulo se trata de comprender y definir las necesidades, objetivos, funcionalidades y actores que van a hacer uso de la herramienta a desarrollar. Para lograrlo se van plantear diferentes requisitos funcionales, no funcionales y casos de uso como solución a las siguientes preguntas.

- ¿Cómo debe comportarse la aplicación?

- ¿Qué flujo tiene cada funcionalidad?

- ¿Qué limitaciones y restricciones existen?

- ¿Qué tipos de usuarios existen?

- ¿Qué acciones puede realizar cada tipo de usuario?

- ¿Cómo satisfacer las necesidades del 
usuario?

- ¿Qué espera el usuario?


%\begin{itemize}
%    \item ¿Cómo debe comportarse la aplicación?
%    \item ¿Qué flujo tiene cada funcionalidad?
%    \item ¿Qué limitaciones y restricciones existen?
%    \item ¿Qué tipos de usuarios existen?
%    \item ¿Qué acciones puede realizar cada tipo de usuario?
%    \item ¿Cómo satisfacer las necesidades del usuario?
%    \item ¿Qué espera el usuario?
%\end{itemize}


\section{Especificación de requisitos}

\subsection{Requisitos funcionales}

Los requisitos funcionales (R.F.) son aquellos que definen como un sistema o componente tiene que funcionar o comportarse. Estos en este caso,
son los siguientes:

\begin{itemize}
    \item R.F. 1: El sistema permitirá representar información a través de distintas visualizaciones.
    \item R.F. 2: Las visualizaciones serán interactivas cuando el usuario hace click o hover.
    \item R.F. 3: El usuario podrá filtrar los datos de entrada de las visualizaciones.
    \item R.F. 4: El sistema permitirá mostrar información relevante a un paciente específico.
    \item R.F. 5: El sistema utlilizará Inteligencia Artificial para generar resúmenes del historial de un paciente.
    \item R.F. 6: El sistema implementará un chat con Inteligencia Artificial para que el usuario realice preguntas sobre los datos.
    
\end{itemize}

\subsection{Requisitos no funcionales}

Los requisitos no funcionales (R.N.F.) son aquellas restricciones que se imponen o existen en relación a la distinta funcionalidad que existe en la herramienta, es decir, las restricciones que presentan los requisitos funcionales. Estos en este caso, son los siguientes:

\begin{itemize}
    \item R.N.F. 1: El sistema en caso de error, gestionará este de manera interna para no afectar la experiencia del usuario que este consumiendo la herramienta.
    \item R.N.F. 2: La herramienta será compatible con los navegadores más usados en la actualidad, así como su uso en distintos sistemas operativos. 
    \item R.N.F. 3: El código desarrollado tendrá que estar bien modularizado y estructurado para que su mantenimiento sea una tarea lo más sencilla posible.
    
\end{itemize}

\section{Casos de uso}

\subsection{Actores}

\subsection{Diagramas}

% actores, casos de uso, diagramas de actividad , etc ..... ??? 