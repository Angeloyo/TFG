\chapter{Introducción}

En este primer capítulo se expone el contexto y motivación del proyecto, los objetivos planteados, y la estructura que se seguirá en la memoria.

\section{Contexto y motivación}

La transformación digital del sector sanitario ha avanzado de forma decidida en las últimas décadas. En Estados Unidos, por ejemplo, cerca del 96\% de los hospitales ya habían adoptado sistemas de historiales clínicos electrónicos (EHR) para 2015 \cite{Henry2016_EHR}. Este crecimiento ha generado un volumen masivo de datos retrospectivos cruciales para la investigación \cite{Halevy2009_data}, pero ha traído consigo desafíos significativos: gran parte de la información permanece en formatos poco estructurados y su análisis requiere habilidades técnicas avanzadas. En este escenario, el éxito de la base de datos MIMIC-III (Medical Information Mart for Intensive Care) demostró el valor de un modelo de acceso permisivo a datos de-identificados, habilitando cientos de estudios que mejoraron la práctica clínica \cite{MIMICIII_paper}. El proyecto MIMIC-IV \cite{MIMICIV_paper, MIMICIV_dataset}, sobre el que se fundamenta este trabajo, nace para continuar y ampliar ese legado.

Este conjunto de datos de acceso abierto, alojado en la plataforma de investigación PhysioNet \cite{PhysioNet_paper}, contiene información clínica detallada y anonimizada del Beth Israel Deaconess Medical Center de Boston. Concretamente, en su versión 3.1, abarca los datos de más de 364.000 pacientes, correspondientes a casi 550.000 admisiones hospitalarias, de las cuales se derivan más de 94.000 estancias en Unidades de Cuidados Intensivos (UCI). La granularidad de esta información, que abarca desde datos demográficos y resultados de laboratorio hasta procedimientos médicos y evolución clínica, facilita la investigación en todo tipo de áreas.

Sin embargo, a pesar de este inmenso potencial para la investigación, su aprovechamiento práctico por parte de los profesionales sanitarios sin conocimientos informáticos avanzados sigue siendo limitado. La motivación fundamental de este proyecto es derribar esa barrera, desarrollando una plataforma que aproveche las últimas tecnologías en Inteligencia Artificial. El objetivo es transformar los datos crudos en conocimiento clínico accionable, permitiendo a los usuarios sin perfil técnico no solo visualizar la información, sino también descubrir patrones complejos y obtener el máximo rendimiento de este valioso conjunto de datos para acelerar la generación de hipótesis y la toma de decisiones.

\section{Objetivos}

\subsection{Objetivo general}

Este proyecto tiene como objetivo principal desarrollar una aplicación web con Inteligencia Artificial que permita visualizar, analizar, predecir y explicar historiales clínicos contenidos en MIMIC-IV.

\subsection{Objetivos específicos}

\begin{itemize}
    \item \textbf{Estudio y adaptación del conjunto de datos MIMIC-IV:} Analizar la estructura del dataset y optimizar su almacenamiento en una base de datos para garantizar consultas eficientes.
    \item \textbf{Diseño e implementación de una API RESTful:} Desarrollar un backend que sirva como intermediario entre la base de datos y el frontend.
    \item \textbf{Desarrollo de una plataforma de visualización interactiva:} Construir una aplicación web con que ofrezca una interfaz intuitiva y visualizaciones dinámicas para facilitar la interpretación de los datos.
    \item \textbf{Implementación de un asistente de Inteligencia Artificial:} Implementar un chat con IA que permita a los usuarios realizar consultas en lenguaje natural sobre la base de datos.
    \item \textbf{Predicción de datos con Inteligencia Artificial} Implementar la predicción de datos futuros a partir de datos pasados para obtener indicadores de riesgo.
    \item \textbf{Despliegue y validación del sistema:} Configurar el entorno de producción y evaluar la usabilidad, el rendimiento y la utilidad de la plataforma final.
\end{itemize}

\section{Estructura de la memoria}

El presente documento se estructura de la siguiente manera:

\begin{itemize}
    \item \textbf{Capítulo 1: Introducción}. Presenta el contexto, motivación, objetivos y la estructura del documento.
    
    \item \textbf{Capítulo 2: Estado del arte}. Revisión de la literatura y trabajos previos.

    ...
    
    %\item \textbf{Capítulo 3: Diseño}. Descripción detallada de la arquitectura, modelos de datos y diseño de interfaces de la aplicación.
    
    %\item \textbf{Capítulo 4: Implementación}. Explicación del proceso de desarrollo de la aplicación, incluyendo tecnologías utilizadas, configuración del entorno y aspectos técnicos relevantes.
    
    %\item \textbf{Capítulo 5: Resultados y evaluación}. Presentación de los resultados obtenidos y evaluación del desempeño de la aplicación en términos de usabilidad, eficiencia y utilidad.
    
    %\item \textbf{Capítulo 6: Conclusiones y trabajo futuro}. Resumen de las contribuciones del proyecto, limitaciones encontradas y posibles líneas de trabajo futuro.
\end{itemize}
