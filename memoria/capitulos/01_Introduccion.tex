\chapter{Introducción}

En este primer capítulo se expone el contexto y motivación del proyecto, los objetivos planteados, y la estructura que se seguirá en la memoria.

\section{Contexto y motivación}

La transformación digital del sector sanitario ha avanzado de forma decidida en las últimas décadas. En Estados Unidos, por ejemplo, cerca del 96\% de los hospitales ya habían adoptado sistemas de historiales clínicos electrónicos (EHR) para 2015 \cite{Henry2016_EHR}. Este avance ha generado un volumen masivo de datos clínicos, fundamentales para la investigación en áreas como la epidemiología o el desarrollo de modelos predictivos \cite{Halevy2009_data}. Sin embargo, aunque la generación de datos es continua y creciente, la mayoría de estos registros están sujetos a estrictas restricciones de acceso debido a cuestiones legales, éticas y, sobre todo, de privacidad del paciente. Esta protección, imprescindible para salvaguardar la confidencialidad, limita el potencial de los datos para la investigación y la innovación clínica.

En este contexto, la aparición de bases de datos como MIMIC-III \cite{MIMICIII_paper} (Medical Information Mart for Intensive Care) supuso un hito al demostrar que es posible compartir datos clínicos reales de forma segura mediante procesos rigurosos de desidentificación, permitiendo así la realización de cientos de estudios que han mejorado la práctica clínica \cite{Kallout2025_contribution}. El proyecto MIMIC-IV \cite{MIMICIV_paper, MIMICIV_dataset}, sobre el que se fundamenta este trabajo, continúa y amplía ese legado con información clínica detallada y anonimizada del Beth Israel Deaconess Medical Center de Boston. En su versión 3.1, MIMIC-IV incluye datos de más de 364.000 pacientes y casi 550.000 admisiones hospitalarias, abarcando desde datos demográficos y resultados de laboratorio hasta procedimientos médicos y evolución clínica.

\newpage
No obstante, a pesar de la riqueza y el potencial de MIMIC-IV, persiste un reto relevante: extraer conocimiento útil de estos datos sigue siendo complejo para quienes no poseen formación técnica avanzada. La barrera tecnológica impide que muchos profesionales sanitarios, investigadores clínicos y estudiantes puedan aprovechar plenamente la información disponible. Ante esta situación, la motivación fundamental de este proyecto es precisamente derribar ese obstáculo, desarrollando una plataforma donde cualquier usuario pueda visualizar, explorar y descubrir patrones complejos en los datos, acelerando así la generación de hipótesis y la toma de decisiones fundamentadas en evidencia real.

\section{Objetivos}

\subsection{Objetivo general}

Este proyecto tiene como objetivo principal desarrollar una aplicación web con Inteligencia Artificial que permita visualizar, analizar, predecir y explicar historiales clínicos contenidos en MIMIC-IV.

\subsection{Objetivos específicos}

\begin{itemize}
\item \textbf{OE1.} Almacenar el conjunto de datos MIMIC-IV en una base de datos para poder realizar consultas eficientes.
\item \textbf{OE2.} Desarrollar un backend API RESTful que sirva como intermediario entre la base de datos y el frontend.
\item \textbf{OE3.} Construir una plataforma web con interfaz intuitiva y visualizaciones dinámicas para facilitar la interpretación de los datos.
\item \textbf{OE4.} Implementar un asistente de IA que permita realizar consultas en lenguaje natural sobre la base de datos.
\item \textbf{OE5.} Implementar la predicción de datos con IA para obtener indicadores de riesgo.
\item \textbf{OE6.} Configurar el entorno de producción y evaluar la usabilidad, el rendimiento y la utilidad de la plataforma final.
\end{itemize}

\newpage
\section{Estructura de la memoria}

El presente documento se estructura de la siguiente manera:

\begin{itemize}
    \item \textbf{Capítulo 1: Introducción}. Presenta el contexto, motivación, objetivos y la estructura del documento.
    
    \item \textbf{Capítulo 2: Estado del arte}. Revisión de la literatura y trabajos previos.

    ...
    
    %\item \textbf{Capítulo 3: Diseño}. Descripción detallada de la arquitectura, modelos de datos y diseño de interfaces de la aplicación.
    
    %\item \textbf{Capítulo 4: Implementación}. Explicación del proceso de desarrollo de la aplicación, incluyendo tecnologías utilizadas, configuración del entorno y aspectos técnicos relevantes.
    
    %\item \textbf{Capítulo 5: Resultados y evaluación}. Presentación de los resultados obtenidos y evaluación del desempeño de la aplicación en términos de usabilidad, eficiencia y utilidad.
    
    %\item \textbf{Capítulo 6: Conclusiones y trabajo futuro}. Resumen de las contribuciones del proyecto, limitaciones encontradas y posibles líneas de trabajo futuro.
\end{itemize}
