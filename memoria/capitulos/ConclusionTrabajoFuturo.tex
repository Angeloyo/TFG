\chapter{Conclusiones y trabajo futuro}

Este proyecto ha desarrollado exitosamente una plataforma web de visualización de datos clínicos que hace accesible el conjunto de datos MIMIC-IV mediante una interfaz intuitiva y funcionalidades de inteligencia artificial. La implementación ha demostrado cómo las tecnologías modernas pueden reducir las barreras técnicas para el análisis de datos sanitarios, facilitando que profesionales sin formación tecnológica avanzada puedan extraer conocimiento valioso de bases de datos clínicas complejas.

\section{Conclusiones}

A continuación se presenta una evaluación detallada del cumplimiento de los objetivos específicos planteados al inicio del proyecto:

\textbf{OE1. Investigar sobre el problema y sus posibles soluciones (100\% completado).} Se realizó una revisión exhaustiva del estado del arte que abarcó desde las bases de datos clínicas abiertas hasta las tecnologías de visualización y los grandes modelos de lenguaje en medicina. Esta investigación, documentada en el Capítulo 2, permitió identificar la carencia de plataformas abiertas para explorar MIMIC-IV de forma accesible y justificó las decisiones tecnológicas adoptadas en el proyecto.

\textbf{OE2. Almacenar el conjunto de datos MIMIC-IV en una base de datos para realizar consultas eficientes (100\% completado).} Se migró exitosamente MIMIC-IV desde archivos CSV a MongoDB, tanto en versión completa como demo. El proceso incluyó la creación de índices optimizados y colecciones auxiliares para mejorar el rendimiento. La implementación se detalla en la Sección 3.1, donde se documenta el proceso de migración, la estructura de datos resultante y las optimizaciones aplicadas.

\textbf{OE3. Desarrollar un backend API RESTful (100\% completado).} Se implementó una API completa con FastAPI que incluye endpoints para pacientes, dashboard, visualizaciones y chat. La arquitectura modular, descrita en la Sección 3.1, facilita el mantenimiento y extensibilidad del sistema. La API está desplegada en producción y funciona de manera estable.

\textbf{OE4. Construir una plataforma web con interfaz intuitiva y visualizaciones dinámicas (100\% completado).} Se desarrolló un frontend en Next.js con TypeScript que incluye seis visualizaciones interactivas diferentes, organizadas en categorías clínicas. La plataforma, documentada en la Sección 3.2, ofrece desde gráficos básicos hasta visualizaciones complejas como diagramas sunburst y chord, todos ellos responsivos y optimizados para la interpretación clínica.

\textbf{OE5. Implementar funcionalidades con IA para consultas en lenguaje natural y resúmenes de pacientes (100\% completado).} Se integró GPT-4.1 mediante el protocolo MCP (Model Context Protocol), permitiendo consultas directas sobre la base de datos en lenguaje natural. Además, se implementó la generación automática de resúmenes de historiales clínicos. Esta funcionalidad, explicada en la Sección 3.1.2, representa una innovación técnica al utilizar uno de los estándares más recientes para la integración de LLMs con sistemas de datos.

\textbf{OE6. Configurar el entorno de producción (100\% completado).} Se desplegó exitosamente la plataforma completa: el backend mediante Cloudflare Tunnels y el frontend en Vercel con CI/CD automatizado desde GitHub. 

\textbf{OE7. Redacción de la memoria (100\% completado).} Se completó la documentación técnica del proyecto, incluyendo contexto teórico, decisiones de implementación y resultados obtenidos, siguiendo las directrices académicas establecidas.



Personalmente, me siento muy satisfecho con los resultados obtenidos en este TFG. Ha sido especialmente gratificante poder combinar mis intereses en tecnología y salud, desarrollando una herramienta que puede tener un impacto real en la democratización del acceso a datos clínicos. El aprendizaje ha sido intenso y continuo: desde dominar nuevas tecnologías hasta comprender la complejidad de los sistemas sanitarios.




El trabajo me ha permitido desarrollar significativamente mis habilidades de resolución de problemas, especialmente al enfrentar desafíos técnicos complejos como la optimización de consultas en bases de datos masivas o la implementación de visualizaciones interactivas avanzadas. También he mejorado mis capacidades de comunicación técnica, tanto en la documentación del código como en la redacción de esta memoria.


\newpage
\section{Trabajo futuro}

A partir del trabajo desarrollado, se identifican múltiples líneas de mejora y extensión que podrían implementarse en el futuro:



\textbf{Integración de módulos adicionales de MIMIC-IV.} Actualmente la plataforma utiliza los módulos \texttt{hosp} e \texttt{icu}. La incorporación de los módulos \texttt{ed} (urgencias), \texttt{note} (notas clínicas) y \texttt{cxr} (radiografías) ampliaría significativamente las posibilidades de análisis. Para el módulo de notas, sería necesario implementar técnicas avanzadas de procesamiento de lenguaje natural, mientras que las radiografías requerirían capacidades de análisis de imágenes médicas.

\textbf{Sistema de filtros temporales y de cohortes.} Implementar funcionalidades que permitan a los usuarios filtrar datos por períodos específicos, tipos de pacientes o condiciones clínicas. Esto requeriría desarrollar una interfaz de filtros avanzada y optimizar las consultas de base de datos para mantener el rendimiento con filtros complejos.

\textbf{Exportación de datos y visualizaciones.} Añadir capacidades para exportar tanto los datos consultados como las visualizaciones generadas en formatos estándar (CSV, PDF, PNG). Esta funcionalidad sería valiosa para investigadores que necesiten integrar los resultados en publicaciones o presentaciones.



%\textbf{Optimización de rendimiento para consultas complejas.} Aunque el sistema funciona correctamente, algunas consultas sobre las colecciones más grandes pueden optimizarse mediante técnicas como la agregación precomputada de estadísticas frecuentes o la implementación de cachés inteligentes para consultas recurrentes.

%\textbf{Sistema de autenticación y perfiles de usuario.} Implementar un sistema de registro y autenticación que permita a los usuarios guardar consultas favoritas, crear dashboards personalizados y mantener un historial de análisis realizados. Esto requeriría diseñar una base de datos adicional para gestionar usuarios y sus preferencias.

%\textbf{API más robusta con documentación interactiva.} Expandir la API actual con más endpoints especializados y generar documentación interactiva completa que facilite su uso por parte de desarrolladores externos que deseen integrar la plataforma en sus propios sistemas.



\textbf{Modelos predictivos especializados.} Desarrollar e integrar modelos de machine learning específicos para predicción de riesgos clínicos, como predicción de mortalidad hospitalaria, duración de estancias o riesgo de readmisión. Estos modelos podrían entrenarse directamente sobre los datos de MIMIC-IV y ofrecer predicciones explicables.

%\textbf{Asistente de análisis estadístico.} Extender las capacidades de IA para que el sistema pueda sugerir automáticamente análisis estadísticos relevantes basándose en los datos consultados, generar hipótesis de investigación y proponer visualizaciones apropiadas según el tipo de datos.

%\textbf{Integración multimodal.} Si se incorporan las imágenes del módulo CXR, desarrollar capacidades de análisis multimodal que combinen datos estructurados, texto libre y radiografías para generar resúmenes más completos de los pacientes.



%\textbf{Migración a arquitectura de microservicios.} Para facilitar el mantenimiento y escalabilidad del sistema, se podría refactorizar la aplicación hacia una arquitectura de microservicios, separando las funcionalidades de visualización, IA, y acceso a datos en servicios independientes.

\textbf{Soporte para múltiples bases de datos clínicas.} Generalizar la arquitectura para soportar otros conjuntos de datos clínicos además de MIMIC-IV, como eICU, MIMIC-III o bases de datos locales que sigan estándares como OMOP-CDM. Esto requeriría desarrollar adaptadores de datos y abstraer la lógica de acceso a datos.



%\textbf{Plataforma como servicio (SaaS).} El sistema desarrollado podría evolucionar hacia una plataforma comercial que ofrezca servicios de análisis de datos clínicos para hospitales e instituciones de investigación. Esto requeriría implementar funcionalidades empresariales como facturación, multi-tenancy y acuerdos de nivel de servicio.

\textbf{Marketplace de visualizaciones.} Crear un ecosistema donde desarrolladores puedan contribuir con nuevas visualizaciones y análisis, estableciendo una comunidad activa alrededor de la plataforma. Esto podría incluir un sistema de plugins y una tienda de extensiones.

El trabajo desarrollado sienta las bases sólidas para todas estas líneas de desarrollo futuro, habiendo demostrado la viabilidad técnica y el valor potencial de hacer accesibles los datos clínicos complejos a través de interfaces modernas e intuitivas.

